\section{Sección}
\subsection{Subsección}
\subsubsection{Subsubsección}

Rick astley \ref{fig:Rick1} and Ricks Astleys \ref{fig:Rick2}.
\begin{figure}[H]
    \centering
    \captionsetup{justification=centering}
    \includegraphics[width=6cm]{images/rick.jpg}
    \caption{Rick.}
    \label{fig:Rick1}
\end{figure}

\begin{figure}[H]
    \centering
    \captionsetup{justification=centering,margin=2cm}
    \subfigure[Rick.]{\includegraphics[width=6cm]{images/rick.jpg}}
    \subfigure[Rick.]{\includegraphics[width=6cm]{images/rick.jpg}}
    \subfigure[Rick.]{\includegraphics[width=6cm]{images/rick.jpg}}
    \subfigure[Rick.]{\includegraphics[width=6cm]{images/rick.jpg}}
    \caption{Rick.}
    \label{fig:Rick2}
\end{figure}

Rick cite \cite{book} and Rick cite \cite{online}
\clearpage
%%%%%%%%%%%%%%%%%%%%%%%%%%%%%%%%%%
\subsection{Sección 2}
\begin{tcolorbox}
[colback=white!5!white,colframe=green!75!black,fonttitle=\bfseries,title=Cuadro]
Había una vez truz
\end{tcolorbox}

\begin{tcolorbox}
[colback=white!5!white,colframe=blue!75!black,fonttitle=\bfseries,title=Cuadro con código]
\begin{minted}[tabsize=2,breaklines, fontsize=\scriptsize, linenos = False]{python}
def suma(a, b):
    """
    Esta función toma dos números como entrada y devuelve la suma de ellos.
    
    Args:
    a (float o int): El primer número.
    b (float o int): El segundo número.
    
    Returns:
    float o int: La suma de los dos números.
    """
    resultado = a + b
    return resultado

# Ejemplo de uso de la función
num1 = 5
num2 = 7
resultado_suma = suma(num1, num2)
print("El resultado de la suma es:", resultado_suma)

\end{minted}
\end{tcolorbox}





\href{https://youtu.be/mCdA4bJAGGk}{Texto con Hipervínculo.}


\clearpage